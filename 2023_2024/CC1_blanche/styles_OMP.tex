%\usepackage[usenames, dvipsnames]{xcolor}
%\usepackage[utf8x]{inputenc}
\usepackage[T1]{fontenc}
\usepackage{tikz}
\usepackage{pgfplots}
\usetikzlibrary{calc,arrows,fadings,decorations.pathreplacing,decorations.markings,patterns,shapes.geometric}
\usepackage{calc}	    	% Faire des calculs sur les longueurs.
\usepackage{ifthen}		
\usepackage{fp}
\usepackage{verbatim}
\usepackage[active,tightpage]{preview}
%\PreviewEnvironment{tikzpicture}
\setlength\PreviewBorder{5pt}

%------ mes couleurs -------
\definecolor{fond}{rgb}{1,0.9,0.75}
\definecolor{filet}{rgb}{1,0.6,0}
\definecolor{monOrange}{RGB}{255,157,0}
\definecolor{monBleu}{rgb}{0.2,0.4,0.6}
% \definecolor{monBleu}{RGB}{54,134,193}
\definecolor{monCyan}{RGB}{74,181,247}
\definecolor{monGris}{RGB}{100,100,100}

%------ styles tikz ---------------
\colorlet{darkblue}{blue!50!black} 
\tikzset{>=stealth,inner sep=0pt, outer sep=2pt,}
\tikzset{tiret/.style={gray,dashed}}
\tikzset{doublefleche/.style={|<->|,>=stealth,thin}}
\tikzset{titre/.style={inner sep=0pt, outer sep=0pt,above right,text justified,fill=orange!50}}
\tikzset{bloc/.style={rounded corners=4pt,color=white,ball color=lightgray,smooth}}
\tikzset{force/.style={->,ultra thick,rounded corners=4pt,color=monBleu,smooth,line cap=round}}
\tikzset{vecteur/.style={->,thick,color=black,smooth}}
\tikzset{verre/.style={draw=SkyBlue,fill=SkyBlue!30}}
\tikzset{axis/.style={thin,gray}}
\tikzset{figure/.style={thick,color=#1,fill=purple, opacity=0.5}}
\tikzset{ressort/.style={very thick,black,smooth}}
\tikzset{eau/.style={draw=black,fill=blue,opacity=0.5}}
\tikzset{rayon/.style={draw=red!66,thick,line join=round}}
%-----------------------------------

%-------- patatoide ----------
\newcommand{\patate}[1][fill=white] 
{\draw [#1][preaction={fill=white}] (0,0) .. controls +(0.1,0.2) and +(0.3,0.3) .. (1,0) .. controls +(-0.3,-0.3) and +(-0.05,0.15) .. (1,-1) .. controls +(0.2,-0.6) and +(0.1,-0.2) .. (0,-1) .. controls +(-0.1,0.15) and +(-0.2,-0.4) .. (0,0);}
%------ aimant -------
\newcommand{\aimant}[1][ultra thick]{
\draw[fill, color=red,#1] (0,0.2) rectangle(1,-0.2) node[color=white,midway]{\tiny S};
\draw[fill, color=black,#1] (1,0.2) rectangle(2,-0.2) node[color=white,midway]{\tiny N};}

% ------ \arcdecercle{}--- longueur d'arc
\newlength{\longueurarcdecercle}
\newcommand{\arcdecercle}[1]{\ensuremath{%
    \settowidth{\longueurarcdecercle}{\ensuremath{#1}}%
    \overset{%
        \mbox{%
                \resizebox{\longueurarcdecercle-2pt}{3pt}{%
                        \rotatebox{90}{\ensuremath{\hskip-1pt)}}%
                }%
        }%
    }{\text{#1}}%
}}

% commande \resistanceH{option}{nom}  : résistance de longueur unité, centrée en (0,0).
\newcommand{\resistanceH}[2]
{
\draw[#1,fill=white] (-0.5,3pt)--++(1,0)--++(0,-6pt)--++(-1,0)--cycle;
\draw[#1] (0,0) node[above=3pt]{\footnotesize #2};
}
% commande \resistanceV{option}{nom}  : résistance de longueur unité, centrée en (0,0).
\newcommand{\resistanceV}[2]
{
\draw[#1,fill=white] (-3pt,0.5)--++(0,-1)--++(6pt,0)--++(0,1)--cycle;
\draw[#1] (0,0) node[left=3pt]{\footnotesize #2};
}
% \condoH{options}{nom}{valeur} : condensateur horizontal de longueur unité, centrée en (0,0)
\newcommand{\condoH}[3]%
{
\fill[#1,white] (-3pt,-0.5)rectangle(3pt,0.5);
\draw[#1,ultra thick] (-3pt,-0.3)--++(0,.6);
\draw[#1,ultra thick] (3pt,-0.3)--++(0,.6);
\draw[#1] (0,-0.3) node[below]{\footnotesize #3};
\draw[#1] (0,0.3) node[above]{\footnotesize#2};
}

% \condoV{options}{nom}{valeur} : condensateur vertical de longueur unité, centrée en (0,0)
\newcommand{\condoV}[3]%
{ 
\fill[#1,white] (-0.5,-3pt)rectangle(0.5,3pt);
\draw[#1,ultra thick] (-0.3,-3pt)--++(.6,0);
\draw[#1,ultra thick] (-0.3,3pt)--++(.6,0);
\draw[#1] (0.3,0) node[right]{\footnotesize #3};
\draw[#1] (-0.3,0) node[left]{\footnotesize #2};
}

% \bobineH{options}{nom}{valeur} : bobine horizontal de longueur unité, centrée en (0,0)
\newcommand{\bobineH}[3]%
{
\fill[#1,white] (-5mm,-3pt) rectangle (5mm,3pt);
\draw[#1,fill=white,decoration={coil,aspect=0.5,segment length=2mm,amplitude=2mm},decorate] (-0.5,0)--++(1,0)node[midway,above=6pt]{\footnotesize #2} node[midway,below=5pt]{\footnotesize #3};
}



     

